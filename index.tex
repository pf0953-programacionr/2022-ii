% Options for packages loaded elsewhere
\PassOptionsToPackage{unicode}{hyperref}
\PassOptionsToPackage{hyphens}{url}
\PassOptionsToPackage{dvipsnames,svgnames,x11names}{xcolor}
%
\documentclass[
  letterpaper,
  DIV=11,
  numbers=noendperiod]{scrreprt}

\usepackage{amsmath,amssymb}
\usepackage{lmodern}
\usepackage{iftex}
\ifPDFTeX
  \usepackage[T1]{fontenc}
  \usepackage[utf8]{inputenc}
  \usepackage{textcomp} % provide euro and other symbols
\else % if luatex or xetex
  \usepackage{unicode-math}
  \defaultfontfeatures{Scale=MatchLowercase}
  \defaultfontfeatures[\rmfamily]{Ligatures=TeX,Scale=1}
\fi
% Use upquote if available, for straight quotes in verbatim environments
\IfFileExists{upquote.sty}{\usepackage{upquote}}{}
\IfFileExists{microtype.sty}{% use microtype if available
  \usepackage[]{microtype}
  \UseMicrotypeSet[protrusion]{basicmath} % disable protrusion for tt fonts
}{}
\makeatletter
\@ifundefined{KOMAClassName}{% if non-KOMA class
  \IfFileExists{parskip.sty}{%
    \usepackage{parskip}
  }{% else
    \setlength{\parindent}{0pt}
    \setlength{\parskip}{6pt plus 2pt minus 1pt}}
}{% if KOMA class
  \KOMAoptions{parskip=half}}
\makeatother
\usepackage{xcolor}
\setlength{\emergencystretch}{3em} % prevent overfull lines
\setcounter{secnumdepth}{5}
% Make \paragraph and \subparagraph free-standing
\ifx\paragraph\undefined\else
  \let\oldparagraph\paragraph
  \renewcommand{\paragraph}[1]{\oldparagraph{#1}\mbox{}}
\fi
\ifx\subparagraph\undefined\else
  \let\oldsubparagraph\subparagraph
  \renewcommand{\subparagraph}[1]{\oldsubparagraph{#1}\mbox{}}
\fi

\usepackage{color}
\usepackage{fancyvrb}
\newcommand{\VerbBar}{|}
\newcommand{\VERB}{\Verb[commandchars=\\\{\}]}
\DefineVerbatimEnvironment{Highlighting}{Verbatim}{commandchars=\\\{\}}
% Add ',fontsize=\small' for more characters per line
\usepackage{framed}
\definecolor{shadecolor}{RGB}{241,243,245}
\newenvironment{Shaded}{\begin{snugshade}}{\end{snugshade}}
\newcommand{\AlertTok}[1]{\textcolor[rgb]{0.68,0.00,0.00}{#1}}
\newcommand{\AnnotationTok}[1]{\textcolor[rgb]{0.37,0.37,0.37}{#1}}
\newcommand{\AttributeTok}[1]{\textcolor[rgb]{0.40,0.45,0.13}{#1}}
\newcommand{\BaseNTok}[1]{\textcolor[rgb]{0.68,0.00,0.00}{#1}}
\newcommand{\BuiltInTok}[1]{\textcolor[rgb]{0.00,0.23,0.31}{#1}}
\newcommand{\CharTok}[1]{\textcolor[rgb]{0.13,0.47,0.30}{#1}}
\newcommand{\CommentTok}[1]{\textcolor[rgb]{0.37,0.37,0.37}{#1}}
\newcommand{\CommentVarTok}[1]{\textcolor[rgb]{0.37,0.37,0.37}{\textit{#1}}}
\newcommand{\ConstantTok}[1]{\textcolor[rgb]{0.56,0.35,0.01}{#1}}
\newcommand{\ControlFlowTok}[1]{\textcolor[rgb]{0.00,0.23,0.31}{#1}}
\newcommand{\DataTypeTok}[1]{\textcolor[rgb]{0.68,0.00,0.00}{#1}}
\newcommand{\DecValTok}[1]{\textcolor[rgb]{0.68,0.00,0.00}{#1}}
\newcommand{\DocumentationTok}[1]{\textcolor[rgb]{0.37,0.37,0.37}{\textit{#1}}}
\newcommand{\ErrorTok}[1]{\textcolor[rgb]{0.68,0.00,0.00}{#1}}
\newcommand{\ExtensionTok}[1]{\textcolor[rgb]{0.00,0.23,0.31}{#1}}
\newcommand{\FloatTok}[1]{\textcolor[rgb]{0.68,0.00,0.00}{#1}}
\newcommand{\FunctionTok}[1]{\textcolor[rgb]{0.28,0.35,0.67}{#1}}
\newcommand{\ImportTok}[1]{\textcolor[rgb]{0.00,0.46,0.62}{#1}}
\newcommand{\InformationTok}[1]{\textcolor[rgb]{0.37,0.37,0.37}{#1}}
\newcommand{\KeywordTok}[1]{\textcolor[rgb]{0.00,0.23,0.31}{#1}}
\newcommand{\NormalTok}[1]{\textcolor[rgb]{0.00,0.23,0.31}{#1}}
\newcommand{\OperatorTok}[1]{\textcolor[rgb]{0.37,0.37,0.37}{#1}}
\newcommand{\OtherTok}[1]{\textcolor[rgb]{0.00,0.23,0.31}{#1}}
\newcommand{\PreprocessorTok}[1]{\textcolor[rgb]{0.68,0.00,0.00}{#1}}
\newcommand{\RegionMarkerTok}[1]{\textcolor[rgb]{0.00,0.23,0.31}{#1}}
\newcommand{\SpecialCharTok}[1]{\textcolor[rgb]{0.37,0.37,0.37}{#1}}
\newcommand{\SpecialStringTok}[1]{\textcolor[rgb]{0.13,0.47,0.30}{#1}}
\newcommand{\StringTok}[1]{\textcolor[rgb]{0.13,0.47,0.30}{#1}}
\newcommand{\VariableTok}[1]{\textcolor[rgb]{0.07,0.07,0.07}{#1}}
\newcommand{\VerbatimStringTok}[1]{\textcolor[rgb]{0.13,0.47,0.30}{#1}}
\newcommand{\WarningTok}[1]{\textcolor[rgb]{0.37,0.37,0.37}{\textit{#1}}}

\providecommand{\tightlist}{%
  \setlength{\itemsep}{0pt}\setlength{\parskip}{0pt}}\usepackage{longtable,booktabs,array}
\usepackage{calc} % for calculating minipage widths
% Correct order of tables after \paragraph or \subparagraph
\usepackage{etoolbox}
\makeatletter
\patchcmd\longtable{\par}{\if@noskipsec\mbox{}\fi\par}{}{}
\makeatother
% Allow footnotes in longtable head/foot
\IfFileExists{footnotehyper.sty}{\usepackage{footnotehyper}}{\usepackage{footnote}}
\makesavenoteenv{longtable}
\usepackage{graphicx}
\makeatletter
\def\maxwidth{\ifdim\Gin@nat@width>\linewidth\linewidth\else\Gin@nat@width\fi}
\def\maxheight{\ifdim\Gin@nat@height>\textheight\textheight\else\Gin@nat@height\fi}
\makeatother
% Scale images if necessary, so that they will not overflow the page
% margins by default, and it is still possible to overwrite the defaults
% using explicit options in \includegraphics[width, height, ...]{}
\setkeys{Gin}{width=\maxwidth,height=\maxheight,keepaspectratio}
% Set default figure placement to htbp
\makeatletter
\def\fps@figure{htbp}
\makeatother
\newlength{\cslhangindent}
\setlength{\cslhangindent}{1.5em}
\newlength{\csllabelwidth}
\setlength{\csllabelwidth}{3em}
\newlength{\cslentryspacingunit} % times entry-spacing
\setlength{\cslentryspacingunit}{\parskip}
\newenvironment{CSLReferences}[2] % #1 hanging-ident, #2 entry spacing
 {% don't indent paragraphs
  \setlength{\parindent}{0pt}
  % turn on hanging indent if param 1 is 1
  \ifodd #1
  \let\oldpar\par
  \def\par{\hangindent=\cslhangindent\oldpar}
  \fi
  % set entry spacing
  \setlength{\parskip}{#2\cslentryspacingunit}
 }%
 {}
\usepackage{calc}
\newcommand{\CSLBlock}[1]{#1\hfill\break}
\newcommand{\CSLLeftMargin}[1]{\parbox[t]{\csllabelwidth}{#1}}
\newcommand{\CSLRightInline}[1]{\parbox[t]{\linewidth - \csllabelwidth}{#1}\break}
\newcommand{\CSLIndent}[1]{\hspace{\cslhangindent}#1}

\KOMAoption{captions}{tableheading}
\makeatletter
\makeatother
\makeatletter
\@ifpackageloaded{bookmark}{}{\usepackage{bookmark}}
\makeatother
\makeatletter
\@ifpackageloaded{caption}{}{\usepackage{caption}}
\AtBeginDocument{%
\ifdefined\contentsname
  \renewcommand*\contentsname{Table of contents}
\else
  \newcommand\contentsname{Table of contents}
\fi
\ifdefined\listfigurename
  \renewcommand*\listfigurename{List of Figures}
\else
  \newcommand\listfigurename{List of Figures}
\fi
\ifdefined\listtablename
  \renewcommand*\listtablename{List of Tables}
\else
  \newcommand\listtablename{List of Tables}
\fi
\ifdefined\figurename
  \renewcommand*\figurename{Figure}
\else
  \newcommand\figurename{Figure}
\fi
\ifdefined\tablename
  \renewcommand*\tablename{Table}
\else
  \newcommand\tablename{Table}
\fi
}
\@ifpackageloaded{float}{}{\usepackage{float}}
\floatstyle{ruled}
\@ifundefined{c@chapter}{\newfloat{codelisting}{h}{lop}}{\newfloat{codelisting}{h}{lop}[chapter]}
\floatname{codelisting}{Listing}
\newcommand*\listoflistings{\listof{codelisting}{List of Listings}}
\makeatother
\makeatletter
\@ifpackageloaded{caption}{}{\usepackage{caption}}
\@ifpackageloaded{subcaption}{}{\usepackage{subcaption}}
\makeatother
\makeatletter
\@ifpackageloaded{tcolorbox}{}{\usepackage[many]{tcolorbox}}
\makeatother
\makeatletter
\@ifundefined{shadecolor}{\definecolor{shadecolor}{rgb}{.97, .97, .97}}
\makeatother
\makeatletter
\makeatother
\ifLuaTeX
  \usepackage{selnolig}  % disable illegal ligatures
\fi
\IfFileExists{bookmark.sty}{\usepackage{bookmark}}{\usepackage{hyperref}}
\IfFileExists{xurl.sty}{\usepackage{xurl}}{} % add URL line breaks if available
\urlstyle{same} % disable monospaced font for URLs
\hypersetup{
  pdftitle={PF 0953 Programación en R 2022-II},
  pdfauthor={Manuel Vargas},
  colorlinks=true,
  linkcolor={blue},
  filecolor={Maroon},
  citecolor={Blue},
  urlcolor={Blue},
  pdfcreator={LaTeX via pandoc}}

\title{PF 0953 Programación en R 2022-II}
\author{Manuel Vargas}
\date{2022-08-21}

\begin{document}
\maketitle
\ifdefined\Shaded\renewenvironment{Shaded}{\begin{tcolorbox}[boxrule=0pt, borderline west={3pt}{0pt}{shadecolor}, frame hidden, enhanced, interior hidden, breakable, sharp corners]}{\end{tcolorbox}}\fi

\renewcommand*\contentsname{Table of contents}
{
\hypersetup{linkcolor=}
\setcounter{tocdepth}{2}
\tableofcontents
}
\bookmarksetup{startatroot}

\hypertarget{bienvenida}{%
\chapter*{Bienvenida}\label{bienvenida}}
\addcontentsline{toc}{chapter}{Bienvenida}

Este curso trata sobre el manejo, visualización y análisis de datos
geoespaciales mediante el lenguaje de programación
\href{https://www.r-project.org/}{R}. Se imparte en la
\href{https://www.sep.ucr.ac.cr/posgrados/geografia/folleto/maestria_academica_recurso_hidrico.pdf}{Maestría
Académica en Gestión Integrada del Recurso Hídrico para Latinoamérica y
El Caribe} de la \href{https://www.ucr.ac.cr/}{Universidad de Costa
Rica}.

Se estudian los fundamentos de R, sus bibliotecas geoespaciales y sus
capacidades para generar gráficos estadísticos. También se utilizan
herramientas para facilitar la reproducibilidad de los procedimientos y
su comunicación a través de la Internet y otros medios.

El enfoque del curso es teórico-práctico, con lecciones teóricas
combinadas con ejercicios de programación en los cuales los estudiantes
aplican en diversos escenarios de procesamiento de datos los
conocimientos y habilidades aprendidas.

Este sitio web corresponde al curso impartido durante el segundo lectivo
de 2022. Para más información sobre los contenidos, metodología,
evaluación y otros temas, puede consultar el
\href{https://github.com/pf0953-programacionr/2022-ii/blob/main/programa-curso/pf0953-programacionr-g001-2022-ii.pdf}{programa
del curso}.

\textbf{Información de contacto}

Si tiene alguna pregunta o comentario sobre este curso, por favor
contacte a:

\begin{quote}
Manuel Vargas - manuel.vargas\_d@ucr.ac.cr\\
Profesor\\
Universidad de Costa Rica,\\
Ciudad Universitaria Rodrigo Facio,\\
San Pedro de Montes de Oca,\\
Costa Rica.
\end{quote}

Los contenidos de este curso, a menos que se especifique de otra forma,
se comparten mediante una licencia de Creative Commons
Reconocimiento-CompartirIgual 4.0 Internacional.

\part{I - Introducción}

\hypertarget{introducciuxf3n-a-la-ciencia-de-datos-geoespaciales}{%
\chapter{Introducción a la ciencia de datos
geoespaciales}\label{introducciuxf3n-a-la-ciencia-de-datos-geoespaciales}}

\hypertarget{trabajo-previo}{%
\section{Trabajo previo}\label{trabajo-previo}}

\hypertarget{lecturas}{%
\subsection{Lecturas}\label{lecturas}}

Bartomeus Lab. (2016). \emph{A reproducible workflow}.
\url{https://www.youtube.com/watch?v=s3JldKoA0zw}

FOSS4G. (2021). \emph{FOSS4G2021---Open source for open spatial data
science---Anita Graser}.
\url{https://www.youtube.com/watch?v=ZjXb53pOor0}

Krugman, P. (2013). Opinion \textbar{} The Excel Depression. \emph{The
New York Times}.
\url{https://www.nytimes.com/2013/04/19/opinion/krugman-the-excel-depression.html}

Peng, R. D. (2011). Reproducible Research in Computational Science.
\emph{Science, 334}(6060), 1226-1227.
\url{https://doi.org/10.1126/science.1213847}

Singleton, A. D., Spielman, S., \& Brunsdon, C. (2016). Establishing a
framework for Open Geographic Information science. \emph{International
Journal of Geographical Information Science, 30}(8), 1507-1521.
\url{https://doi.org/10.1080/13658816.2015.1137579}

Wu, Q. (2021, octubre 25). A streamlit app for creating timelapse of
annual Landsat imagery (1984--2021). Medium.
\url{https://giswqs.medium.com/a-streamlit-app-for-creating-timelapse-of-annual-landsat-imagery-1984-2021-3db407a8ac32}

\hypertarget{el-componente-geoespacial-de-los-datos}{%
\section{El componente geoespacial de los
datos}\label{el-componente-geoespacial-de-los-datos}}

Una gran parte de los datos disponibles contiene algún tipo de
componente geográfico o espacial \footnote{El adjetivo \emph{geográfico}
  se refiere a la superficie de la Tierra. Así, por ejemplo, las
  \emph{coordenadas geográficas} se utilizan para ubicar cualquier punto
  en la superficie terrestre. El término \emph{espacial} se emplea para
  referirse a cualquier espacio, no siempre localizable en el planeta
  Tierra. En muchas ocasiones, ambas palabras son intercambiables. Por
  ejemplo, muchos de los métodos utilizados para analizar datos
  geográficos pueden aplicarse también en espacios no geográficos como,
  por ejemplo, otros planetas, el cosmos, el cuerpo humano (ej. con
  radiografías) o secuencias genómicas. En los últimos años, se ha
  incrementado el uso del término \emph{geoespacial}, como una forma de
  referirse al subconjunto del espacio correspondiente a la superficie
  de la Tierra (Longley et al. 2005).}. Este componente puede expresarse
de varias formas. Por ejemplo:

\begin{itemize}
\tightlist
\item
  \textbf{Con nombres de lugares}: \emph{El
  \href{https://es.wikipedia.org/wiki/Incilius_periglenes}{sapo dorado
  (Incilius
  periglenes}}\href{https://es.wikipedia.org/wiki/Incilius_periglenes}{)}
  era una especie de anfibio, endémica de los bosques nubosos de altitud
  de Monteverde, Costa Rica.
\item
  \textbf{Con direcciones}: \emph{La
  \href{https://es.wikipedia.org/wiki/Sede_de_la_Organizaci\%C3\%B3n_de_las_Naciones_Unidas}{sede
  de la Organización de las Naciones Unidas (ONU)} está ubicada en la
  ciudad de Nueva York, Estados Unidos, en la Primera Avenida, 750.}
\item
  \textbf{Con coordenadas}: *La cima del
  \href{https://es.wikipedia.org/wiki/Monte_Everest}{Monte Everest} se
  localiza en las coordenadas geográficas 86°55′31″ E y 27°59′17″ N,
  como se muestra en la Figure~\ref{fig-mapa-nepal-everest}.
\end{itemize}

\begin{figure}

{\centering \includegraphics{./img/nepal-map.jpg}

}

\caption{\label{fig-mapa-nepal-everest}Mapa de Nepal que muestra la
ubicación del Monte Everest en el sistema de coordenadas geográficas.
Imagen de \url{https://www.mapsofworld.com/}.}

\end{figure}

Las coordenadas correspondientes a lugares y direcciones pueden
obtenerse a través de un proceso denominado
\href{https://es.wikipedia.org/wiki/Georreferenciaci\%C3\%B3n}{\emph{georreferenciación}},
mediante el cual, en general, se determina la posición espacial de
alguna entidad en un sistema de coordenadas. La georreferenciación puede
emplearse también para obtener las coordenadas de, por ejemplo,
fotografías aéreas o mapas antiguos. Es un proceso que puede resultar
complejo y costoso y para el que se han desarrollado metodologías y
plataformas especializadas (ej.
\href{https://doi.org/10.15468/doc-gg7h-s853}{Chapman AD \& Wieczorek JR
(2020) Georeferencing Best Practices},
\href{https://www.geo-locate.org/}{GEOLocate},
\href{https://nominatim.openstreetmap.org/ui/search.html}{Nominatim}).

En la actualidad, hay una gran cantidad de fuentes que generan datos
georreferenciados (i.e.~ubicados en un sistema de coordenadas). Entre
estas pueden mencionarse las tecnologías de
\href{https://ec.europa.eu/jrc/en/research-topic/earth-observation}{observación
de la Tierra (\emph{Earth Observation})} (ej.
\href{https://es.wikipedia.org/wiki/Imagen_satelital}{imágenes
satelitales}), los dispositivos móviles y los sensores remotos, entre
muchas otras.

Seguidamente, se describen dos enfoques tecnológicos para el
procesamiento de datos geoespaciales: el de los sistemas de información
geográfica y el de ciencia de datos geoespaciales.

\hypertarget{sistemas-de-informaciuxf3n-geogruxe1fica}{%
\section{Sistemas de información
geográfica}\label{sistemas-de-informaciuxf3n-geogruxe1fica}}

A principios de la década de 1960, el geógrafo inglés
\href{https://es.wikipedia.org/wiki/Roger_Tomlinson}{Roger Tomlinson}
desarrolló en Canadá el que se considera el primer sistema de
información geográfica. Se trataba del
\href{https://en.wikipedia.org/wiki/Canada_Geographic_Information_System}{Canada
Geographic Information System (CGIS)} y su objetivo fue manejar los
datos del inventario geográfico canadiense y su análisis para la gestión
del territorio rural. De manera casi simultánea al trabajo de Tomlinson,
surgieron desarrollos similares en Estados Unidos y en el Reino Unido.
El surgimiento de los sistemas de información geográfica no implicó solo
el surgimiento de nuevas herramientas de software, sino también el
desarrollo de técnicas que hasta entonces no habían sido necesarias
(Olaya 2020) como, por ejemplo, la manipulación de nuevos tipos de datos
geométricos (ej. puntos, líneas, polígonos).

En general, un sistema de información geográfica (SIG) maneja datos
georreferenciados y los asocia con datos convencionales (ej. textos,
números), como se muestra en la Figure~\ref{fig-mapa-qgis}.

\begin{figure}

{\centering \includegraphics[width=6.17in,height=\textheight]{./img/mapa-qgis.png}

}

\caption{\label{fig-mapa-qgis}Mapa elaborado en QGIS que muestra la
ubicación de los aeródromos de Costa Rica.}

\end{figure}

Los SIG presentan los datos en capas (\emph{layers}). Por ejemplo, el
mapa de la Figure~\ref{fig-mapa-qgis} contiene una capa base raster (la
que muestra el mar y el continente), una capa de polígonos
correspondiente a las provincias de Costa Rica y una capa de puntos
correspondiente a los aeródromos. A la izquierda puede apreciarse la
lista de esas capas y a la derecha un cuadro con información detallada
sobre uno de los aeródromos.

Los SIG de escritorio (ej.
\href{https://www.esri.com/en-us/arcgis/products/arcgis-desktop/overview}{ArcGIS
Desktop}, \href{https://www.qgis.org/}{QGIS}) son herramientas con
interfaces de usuario muy gráficas e intuitivas, que no requieren de
conocimientos de programación de computadoras y que permiten generar
cartografía de alta calidad. Sin embargo, son poco flexibles y los
resultados que producen son difícilmente
\href{https://es.wikipedia.org/wiki/Reproducibilidad_y_repetibilidad}{reproducibles}.

\hypertarget{ciencia-de-datos-geoespaciales}{%
\section{Ciencia de datos
geoespaciales}\label{ciencia-de-datos-geoespaciales}}

Durante la última década, el uso de SIG se ha complementado con el de
\href{https://es.wikipedia.org/wiki/Ciencia_de_datos}{ciencia de datos},
lo que posibilitado enriquecer la visualización y el análisis de datos
geoespaciales mediante lenguajes de programación como
\href{https://www.python.org/}{Python},
\href{https://www.r-project.org/}{R} o
\href{http://www.ecma-international.org/publications-and-standards/standards/ecma-262/}{JavaScript},
entre otros.

El uso de técnicas de ciencia de datos y de otros campos relacionados
(ej.
\href{https://es.wikipedia.org/wiki/Aprendizaje_autom\%C3\%A1tico}{aprendizaje
automatizado}, \href{https://es.wikipedia.org/wiki/Macrodatos}{\emph{big
data}}) ha permitido aplicar a los datos geoespaciales técnicas y
metodologías como
\href{https://es.wikipedia.org/wiki/An\%C3\%A1lisis_de_la_regresi\%C3\%B3n}{análisis
de regresión} y
\href{https://es.wikipedia.org/wiki/Clasificaci\%C3\%B3n_estad\%C3\%ADstica}{clasificación
estadística}.

\hypertarget{reproducibilidad}{%
\section{Reproducibilidad}\label{reproducibilidad}}

Una de las principales características que distingue al enfoque de
ciencia de datos del enfoque de SIG es la
\href{https://es.wikipedia.org/wiki/Reproducibilidad_y_repetibilidad}{reproducibilidad}.
En general, la reproducibilidad es la capacidad de un ensayo o
experimento de ser reproducido por otros. Más formalmente, en
investigación cuantitativa, un análisis se considera reproducible si
\emph{``el código fuente y los datos utilizados por un investigador para
llegar a un resultado están disponibles y son suficientes para que otro
investigador, trabajando de manera independiente, pueda llegar al mismo
resultado''} (Gandrud 2020).

La reproducibilidad, junto con la
\href{https://es.wikipedia.org/wiki/Falsabilidad}{falsabilidad}, es uno
de los pilares del
\href{https://es.wikipedia.org/wiki/M\%C3\%A9todo_cient\%C3\%ADfico}{método
científico}. Sin embargo, en años recientes, se ha generado una
creciente preocupación debido a que muchos estudios científicos
publicados fallan las pruebas de reproducibilidad (véase, por ejemplo,
\href{https://www.nytimes.com/2013/04/19/opinion/krugman-the-excel-depression.html}{\emph{The
Excel Depression}, de Paul Krugman}), dando lugar a una
\href{https://es.wikipedia.org/wiki/Crisis_de_replicaci\%C3\%B3n}{crisis
de reproducibilidad o replicabilidad} en varias ciencias.

El concepto de reproducibilidad es cada vez más importante debido, entre
otras razones, al aumento exponencial de datos disponibles y a la
aplicación de la programación de computadoras, para procesar estos
datos, por parte de especialistas de muchas disciplinas.

Alex Singleton y otros autores (Singleton, Spielman, and Brunsdon 2016)
han identificado los siguientes retos para la reproducibilidad en
ciencia de datos geoespaciales:

\begin{enumerate}
\def\labelenumi{\arabic{enumi}.}
\tightlist
\item
  Los datos deben ser de dominio público y estar disponibles para los
  investigadores.
\item
  El software utilizado debe ser de código abierto (\emph{open source})
  y estar disponible para ser revisado.
\item
  Siempre que sea posible, los
  \href{https://es.wikipedia.org/wiki/Flujo_de_trabajo}{flujos de
  trabajo} deben ser públicos y con enlaces a los datos, software y
  métodos de análisis, junto con la documentación necesaria.
\item
  El proceso de
  \href{https://es.wikipedia.org/wiki/Revisi\%C3\%B3n_por_pares}{revisión
  por pares (\emph{peer review process})} y la publicación académica
  deben requerir la presentación de un modelo de flujo de trabajo e
  idealmente la disponibilidad de los materiales necesarios para la
  replicación.
\item
  En los casos en los que la reproducibilidad total no sea posible (ej.
  datos sensibles), los investigadores deben esforzarse por incluir
  todos los aspectos que puedan de un marco de trabajo abierto.
\end{enumerate}

En general, el estándar mínimo de reproducibilidad requiere que los
datos y el código fuente estén disponibles para otros investigadores
(Peng 2011). Sin embargo, dependiendo de las circunstancias y recursos
disponibles, existe todo un espectro de posibilidades, que se ilustra en
la Figure~\ref{fig-espectro-reproducibilidad}.

\begin{figure}

{\centering \includegraphics[width=4.6in,height=\textheight]{./img/espectro-reproducibilidad.png}

}

\caption{\label{fig-espectro-reproducibilidad}Espectro de
reproducibilidad. Imagen de
\href{https://www.youtube.com/watch?v=ZjXb53pOor0}{Anita Graser}, con
base en \href{https://doi.org/10.1126/science.1213847}{(Peng, 2001)}.}

\end{figure}

\hypertarget{herramientas-para-facilitar-la-reproducibilidad}{%
\subsection{Herramientas para facilitar la
reproducibilidad}\label{herramientas-para-facilitar-la-reproducibilidad}}

En esta sección se destacan dos tipos de herramientas que en la
actualidad se consideran esenciales para apoyar la reproducibilidad de
una investigación: los lenguajes de marcado y los sistemas de control de
versiones.

La documentación es vital durante todo el ciclo de vida de una
investigación reproducible. Para documentar, se recomienda utilizar
mecanismos estandarizados y abiertos como el
\href{https://es.wikipedia.org/wiki/HTML}{lenguaje de marcado de
hipertexto (HTML, en inglés, \emph{HyperText Markup Language})} o
\href{https://en.wikipedia.org/wiki/Markdown}{Markdown}, con los cuales
pueden crearse documentos mediante editores de texto simples (i.e.~no se
requiere de software propietario), y exportables a varios formatos (ej.
\href{https://es.wikipedia.org/wiki/LaTeX}{LaTeX},
\href{https://es.wikipedia.org/wiki/PDF}{PDF}).

Para dar mantenimiento, tanto al código fuente como a la documentación,
es necesario un sistema de
\href{https://es.wikipedia.org/wiki/Control_de_versiones}{control de
versiones} como \href{https://es.wikipedia.org/wiki/Git}{Git}, el cual
permite llevar el registro de los cambios en archivos y también facilita
el trabajo colaborativo al reunir las modificaciones hechas por varias
personas. Git es usado en varias plataformas que comparten código fuente
(ej. \href{https://github.com/}{GitHub},
\href{https://about.gitlab.com/}{GitLab}) y que ofrecen servicios
relacionados, como hospedaje de sitios web.

\hypertarget{markdown-lenguaje-de-marcado}{%
\chapter{Markdown: lenguaje de
marcado}\label{markdown-lenguaje-de-marcado}}

\hypertarget{trabajo-previo-1}{%
\section{Trabajo previo}\label{trabajo-previo-1}}

\hypertarget{tutoriales}{%
\subsection{Tutoriales}\label{tutoriales}}

\emph{Markdown Tutorial}. (s. f.). Recuperado 19 de marzo de 2022, de
\url{https://www.markdowntutorial.com/}

\hypertarget{otros}{%
\subsection{Otros}\label{otros}}

\begin{itemize}
\tightlist
\item
  Instale en su computadora el
  \href{https://cloud.r-project.org/}{sistema base del lenguaje R} y
  luego el ambiente integrado de desarrollo
  \href{https://www.rstudio.com/products/rstudio/download/\#download}{RStudio
  Desktop}.
\item
  Cree una cuenta gratuita en la plataforma de desarrollo colaborativo
  \href{https://github.com/}{GitHub}.
\end{itemize}

\hypertarget{resumen}{%
\section{Resumen}\label{resumen}}

Markdown es un lenguaje de marcado ligero ampliamente utilizado en
comunicación científica, documentación de programas e investigación
reproducible.

\hypertarget{descripciuxf3n-general}{%
\section{Descripción general}\label{descripciuxf3n-general}}

\href{https://daringfireball.net/projects/markdown/}{Markdown} es un
\href{https://es.wikipedia.org/wiki/Lenguaje_de_marcado}{lenguaje de
marcado} creado en 2004 por John Gruber. Las ``marcas'' se utilizan para
brindar información acerca de la presentación (ej. negritas, itálicas) o
la estructura (ej. títulos, encabezados) de un documento. Se caracteriza
por ser más sencillo de leer y de usar que otros lenguajes de marcado
(ej. \href{https://es.wikipedia.org/wiki/HTML}{Lenguaje de marcado de
Hipertexto o HTML}), por lo que se considera un
\href{https://es.wikipedia.org/wiki/Lenguaje_de_marcas_ligero}{lenguaje
de marcado ligero}. Los documentos escritos en Markdown pueden
exportarse a una gran variedad de formatos (ej. HTML, DOC, PDF, LaTex)
para ser usados en libros, presentaciones o páginas web, entre otros.
Markdown es ampliamente utilizado en comunicación científica,
documentación de programas e investigación reproducible.

\hypertarget{variaciones}{%
\section{Variaciones}\label{variaciones}}

Las variaciones de Markdown, también llamadas \emph{flavors}, son
extensiones o modificaciones de la especificación original. Entre las
más populares están:

\begin{itemize}
\tightlist
\item
  \href{https://rmarkdown.rstudio.com/}{R Markdown}: para el lenguaje R.
\item
  \href{https://help.github.com/en/github/writing-on-github}{GitHub
  Flavored Markdown}: para la plataforma GitHub.
\item
  \href{https://github.com/Python-Markdown/markdown}{Python Markdown}:
  para el lenguaje Python.
\item
  \href{https://pandoc.org/MANUAL.html\#pandocs-markdown}{Pandoc's
  Markdown}: para el programa \href{https://pandoc.org/}{Pandoc} de
  conversión entre formatos.
\item
  \href{https://kramdown.gettalong.org/quickref.html}{Kramdown}: para el
  lenguaje Ruby.
\end{itemize}

Puede verse una lista más extensa en
\url{https://github.com/commonmark/commonmark-spec/wiki/markdown-flavors}.

\hypertarget{sintaxis}{%
\section{Sintaxis}\label{sintaxis}}

La sintaxis de Markdown permite especificar diferentes componentes de un
documento, entre los que están:

\begin{itemize}
\tightlist
\item
  Encabezados.
\item
  Estilos (ej. negritas, itálicas).
\item
  Citas textuales.
\item
  Enlaces a otros documentos (ej. páginas web).
\item
  Imágenes.
\item
  Listas.
\end{itemize}

\hypertarget{encabezados}{%
\subsection{Encabezados}\label{encabezados}}

Pueden definirse seis niveles de encabezados, mediante símbolos de
numeral (\texttt{\#}) antes del texto. El primer nivel es el de tamaño
de texto más grande y el sexto el más pequeño. En la parte izquierda de
la Figure~\ref{fig-md-encabezados} se muestra la sintaxis Markdown de
los encabezados y a la derecha la forma en que se despliegan en un
documento.

\begin{figure}

{\centering \includegraphics[width=1.98in,height=\textheight]{./img/md-encabezados.png}

}

\caption{\label{fig-md-encabezados}Sintaxis de Markdown - encabezados.}

\end{figure}

\hypertarget{ituxe1licas}{%
\subsection{Itálicas}\label{ituxe1licas}}

Se definen con un asterisco (\texttt{*}) antes y después del texto o con
un guión bajo (\texttt{\_}) antes y después del texto.

\begin{figure}

{\centering \includegraphics[width=1.66in,height=\textheight]{./img/md-italica.png}

}

\caption{\label{fig-md-italicas}Sintaxis de Markdown - itálicas.}

\end{figure}

\hypertarget{negritas}{%
\subsection{Negritas}\label{negritas}}

Se definen con dos asteriscos (\texttt{**}) antes y después del texto o
con dos guiones bajos (\texttt{\_\_}) antes y después del texto.

\begin{figure}

{\centering \includegraphics[width=1.78in,height=\textheight]{./img/md-negrita.png}

}

\caption{\label{fig-md-negritas}Sintaxis de Markdown - negritas.}

\end{figure}

\hypertarget{citas-textuales}{%
\subsection{Citas textuales}\label{citas-textuales}}

Se definen con un símbolo de ``mayor que'' (\texttt{\textgreater{}})
antes de cada línea.

\begin{figure}

{\centering \includegraphics[width=2.22in,height=\textheight]{./img/md-cita.png}

}

\caption{\label{fig-md-citas}Sintaxis de Markdown - citas textuales.}

\end{figure}

\hypertarget{enlaces-hipervuxednculos}{%
\subsection{Enlaces (hipervínculos)}\label{enlaces-hipervuxednculos}}

Se definen con paréntesis cuadrados (\texttt{{[}{]}}) seguidos de
paréntesis redondos (\texttt{()}). En los paréntesis cuadrados se coloca
(opcionalmente) el texto del enlace y en los redondos la dirección del
documento.

\begin{figure}

{\centering \includegraphics[width=2.08in,height=\textheight]{./img/md-enlace.png}

}

\caption{\label{fig-md-enlaces}Sintaxis de Markdown - enlaces.}

\end{figure}

\hypertarget{imuxe1genes}{%
\subsection{Imágenes}\label{imuxe1genes}}

Se definen con un signo de admiración de cierre (\texttt{!}), paréntesis
cuadrados (\texttt{{[}{]}}) y paréntesis redondos (\texttt{()}). En los
paréntesis cuadrados se coloca (opcionalmente) un texto alternativo de
la imagen y en los redondos la dirección de la imagen, ya sea local o
remota.

\begin{figure}

{\centering \includegraphics[width=4.03in,height=\textheight]{./img/md-imagen.png}

}

\caption{\label{fig-md-imagenes}Sintaxis de Markdown - imágenes.}

\end{figure}

\hypertarget{listas-numeradas}{%
\subsection{Listas numeradas}\label{listas-numeradas}}

Se definen con números (\texttt{1.\ 2.\ 3.\ ...}) antes de cada
elemento.

\begin{figure}

{\centering \includegraphics[width=1.91in,height=\textheight]{./img/md-lista-numerada.png}

}

\caption{\label{fig-md-listas-numeradas}Sintaxis de Markdown - listas
numeradas.}

\end{figure}

\hypertarget{listas-no-numeradas}{%
\subsection{Listas no numeradas}\label{listas-no-numeradas}}

Se definen con guiones (\texttt{-}) o asteriscos (\texttt{*}) antes de
cada elemento.

\begin{figure}

{\centering \includegraphics[width=1.81in,height=\textheight]{./img/md-lista-no-numerada.png}

}

\caption{\label{fig-md-listas-no-numeradas}Sintaxis de Markdown - listas
no numeradas.}

\end{figure}

\hypertarget{otros-elementos-de-sintaxis}{%
\subsection{Otros elementos de
sintaxis}\label{otros-elementos-de-sintaxis}}

Para conocer otros elementos de la sintaxis de Markdown, se recomienda
revisar en detalle la \href{https://www.markdownguide.org/}{Guía de
referencia de Markdown}.

\hypertarget{ejercicios}{%
\section{Ejercicios}\label{ejercicios}}

\begin{enumerate}
\def\labelenumi{\arabic{enumi}.}
\tightlist
\item
  Cree un documento Markdown llamado \texttt{README.md}, en RStudio, y
  escriba en este un breve perfil académico (\emph{curriculum}
  académico).

  \begin{itemize}
  \tightlist
  \item
    Incluya información como: nombre, fotografía, datos de contacto,
    áreas de interés, carrera, cursos aprobados, publicaciones, etc.
  \item
    Puede usar información ficticia (\textbf{no incluya datos
    confidenciales o sensibles}).
  \item
    Especifique la fuente de las imágenes (y de cualquier otra
    información para la que sea necesario) y no utilice imágenes para
    las que no tiene autorización. Considere utilizar sitios con
    imágenes con licencias abiertas (ej.
    \href{https://commons.wikimedia.org/}{Wikimedia Commons},
    \href{https://unsplash.com/}{Unsplash},
    \href{https://www.freeimages.com/}{FreeImages}).
  \item
    Asegúrese de utilizar los siguientes elementos de sintaxis Markdown:

    \begin{itemize}
    \tightlist
    \item
      Varios niveles de encabezados.
    \item
      Negritas e itálicas.
    \item
      Listas.
    \item
      Enlaces a sitios web.
    \item
      Imágenes (al menos una local y una remota).
    \end{itemize}
  \end{itemize}
\item
  Cree un repositorio en \href{https://github.com/}{GitHub} llamado
  \texttt{perfil-academico} y suba a este el documento que creó en el
  paso 1.
\item
  Cree un sitio web en \href{https://pages.github.com/}{GitHub Pages}
  con el repositorio creado en el paso 2.
\end{enumerate}

\hypertarget{recursos-de-interuxe9s}{%
\section{Recursos de interés}\label{recursos-de-interuxe9s}}

Carrera Arias, F. J. (2020). \emph{How to Install R on Windows, Mac OS
X, and Ubuntu Tutorial}. DataCamp Community.
\url{https://www.datacamp.com/community/tutorials/installing-R-windows-mac-ubuntu}

\emph{Markdown Guide}. (s. f.). Recuperado 10 de abril de 2022, de
\url{https://www.markdownguide.org/}

\hypertarget{git---sistema-de-control-de-versiones}{%
\chapter{Git - sistema de control de
versiones}\label{git---sistema-de-control-de-versiones}}

\hypertarget{trabajo-previo-2}{%
\section{Trabajo previo}\label{trabajo-previo-2}}

\hypertarget{tutoriales-1}{%
\subsection{Tutoriales}\label{tutoriales-1}}

Abba, I. V. (2021). \emph{Git and GitHub Tutorial -- Version Control for
Beginners}. FreeCodeCamp.Org.
\url{https://www.freecodecamp.org/news/git-and-github-for-beginners/}

\hypertarget{otros-1}{%
\subsection{Otros}\label{otros-1}}

Instale en su computadora el sistema de control de versiones
\href{https://git-scm.com/downloads}{Git}.

\hypertarget{resumen-1}{%
\section{Resumen}\label{resumen-1}}

Git es un sistema para administrar versiones de código fuente o, en
general, de cualquier conjunto de archivos.

\hypertarget{descripciuxf3n-general-1}{%
\section{Descripción general}\label{descripciuxf3n-general-1}}

\href{https://git-scm.com/}{Git} es un sistema de
\href{https://es.wikipedia.org/wiki/Control_de_versiones}{control de
versiones} diseñado para ``rastrear'' cambios en el código fuente
durante el proceso de desarrollo de software. Sin embargo, puede ser
utilizado para llevar el control de los cambios en cualquier conjunto de
archivos (ej.
\href{https://guides.github.com/features/wikis/}{documentación},
\href{https://techcrunch.com/2013/10/09/splice-music/}{música}).

Un sistema de control de versiones proporciona, entre otras ventajas:

\begin{itemize}
\tightlist
\item
  La capacidad de recuperar versiones anteriores de los archivos.
\item
  La capacidad de integrar modificaciones efectuadas por varias personas
  en el mismo conjunto de archivos.
\item
  La capacidad de mantener varias ``ramas'' (\emph{branches}) de un
  producto (ej. ``estable'', ``evaluación'', ``inestable'', como en el
  caso de \href{https://www.debian.org/releases/}{Debian Linux},
  \href{https://grass.osgeo.org/download/software/sources/}{GRASS GIS} y
  muchos otros proyectos de software libre).
\item
  Facilidades para mantener redundancia y respaldos de los archivos (ej.
  \href{https://archiveprogram.github.com/}{Programa de respaldos de
  GitHub}). Esta es una facilidad que implementan algunos servicios en
  la nube.
\end{itemize}

Git fue diseñado por Linus Torvalds en 2005 durante del desarrollo del
\emph{kernel} del sistema operativo Linux. Se caracteriza por ser un
\href{https://es.wikipedia.org/wiki/Control_de_versiones_distribuido}{sistema
de control de versiones distribuido}, lo que significa que el código
fuente puede estar alojado en la estación de trabajo de cualquier
miembro del equipo de desarrollo. No se requiere de un repositorio
``central'', pero también se puede trabajar de esa forma.

El protocolo de Git es utilizado en varios sitios que proveen servicios
de alojamiento de software, entre los que están
\href{https://sourceforge.net/}{SourceForge},
\href{https://bitbucket.org/}{Bitbucket},
\href{https://about.gitlab.com/}{GitLab} y
\href{https://github.com/}{GitHub}.

\hypertarget{funcionamiento-de-git}{%
\section{Funcionamiento de Git}\label{funcionamiento-de-git}}

Desde el punto de vista de un usuario de Git (ej. un programador), Git
se utiliza para sincronizar la versión local (i.e.~en la computadora
personal del usuario) de un conjunto de archivos, llamado proyecto o
repositorio, con la versión que está alojada en un sistema remoto (ej.
GitHub). Cada repositorio se almacena en un directorio (carpeta) del
sistema operativo. La sincronización se realiza principalmente a través
de dos operaciones:

\begin{itemize}
\tightlist
\item
  \textbf{\emph{push}}: para ``subir'' al repositorio remoto los cambios
  realizados en el repositorio local. Esta operación se realiza mediante
  el comando \href{https://git-scm.com/docs/git-push}{git push}. Es
  probable que el sistema remoto le solicite al usuario algún tipo de
  autenticación (ej. nombre de usuario y clave).
\item
  \textbf{\emph{pull}}: para ``bajar'' al repositorio local los cambios
  realizados en el repositorio remoto. Esta operación se realiza
  mediante el comando \href{https://git-scm.com/docs/git-pull}{git
  pull}.
\end{itemize}

Las operaciones \emph{push} y \emph{pull} se ilustran en la
Figure~\ref{fig-git-push-pull}.

\begin{figure}

{\centering \includegraphics[width=1.67in,height=\textheight]{./img/git-push-pull.png}

}

\caption{\label{fig-git-push-pull}Operaciones \emph{push} y \emph{pull}.
Imagen de
\href{https://www.coursera.org/learn/reproducible-templates-analysis/lecture/NGbQv/git-and-github-part-1}{Melinda
Higgins}.}

\end{figure}

Antes de un \emph{push}, el usuario debe seleccionar los archivos que
desea subir mediante el comando
\href{https://git-scm.com/docs/git-add}{git add}, el cual pasa los
archivos a un ``área de espera'' (\emph{staging area}). Luego debe
usarse el comando \href{https://git-scm.com/docs/git-commit}{git commit}
para ``guardar'' los cambios pendientes en el área de espera. Cada
\emph{commit} guarda el estado del conjunto de archivos en un momento
específico (\emph{snapshot}).

La relación entre estas operaciones de Git, se ilustra en la
Figure~\ref{fig-git-push-pull-commit}.

\begin{figure}

{\centering \includegraphics[width=4.27in,height=\textheight]{./img/git-push-pull-commit.png}

}

\caption{\label{fig-git-push-pull-commit}Operaciones de Git. Imagen de
\href{https://medium.com/@stevenklavins94/version-control-part-4-c9387cf5b33e}{Steven
Klavins}.}

\end{figure}

En la Figure~\ref{fig-git-stage-commit-push}, se muestra el
funcionamiento de Git mediante una comparación con el procesamiento de
una compra en línea.

\begin{figure}

{\centering \includegraphics[width=2.66in,height=\textheight]{./img/git-stage-commit-push.png}

}

\caption{\label{fig-git-stage-commit-push}Operaciones de Git y compras
en línea. Imagen de
\href{https://www.coursera.org/learn/reproducible-templates-analysis/lecture/NGbQv/git-and-github-part-2}{Melinda
Higgins}.}

\end{figure}

Otras operaciones de Git de uso frecuente son:

\begin{itemize}
\tightlist
\item
  \href{https://git-scm.com/docs/git-config}{git config}: para
  especificar opciones globales de la sesión de Git (ej. nombre del
  usuario, dirección de corre electrónico).
\item
  \href{https://git-scm.com/docs/git-init}{git init}: para inicializar
  un repositorio git.
\item
  \href{https://git-scm.com/docs/git-clone}{git clone}: para clonar
  (i.e.~copiar) un repositorio remoto en la computadora local.
\item
  \href{https://git-scm.com/docs/git-status}{git status}: para revisar
  el estado de los archivos y, por ejemplo, saber cuales deben pasarse
  al área de espera.
\item
  \href{https://git-scm.com/docs/git-log}{git log}: para revisar el
  historial de \emph{commits}.
\item
  \href{https://git-scm.com/docs/git-show}{git show}: para visualizar
  los cambios efectuados en los \emph{commits}.
\item
  \href{https://git-scm.com/docs/git-reset}{git reset}: para regresar al
  estado correspondiente a un \emph{commit} anterior.
\end{itemize}

\hypertarget{ejemplos-de-uso}{%
\section{Ejemplos de uso}\label{ejemplos-de-uso}}

\hypertarget{clonaciuxf3n-de-un-repositorio-remoto-y-sincronizaciuxf3n-de-los-cambios-efectuados-localmente}{%
\subsection{Clonación de un repositorio remoto y sincronización de los
cambios efectuados
localmente}\label{clonaciuxf3n-de-un-repositorio-remoto-y-sincronizaciuxf3n-de-los-cambios-efectuados-localmente}}

Para seguir este ejemplo:

\begin{enumerate}
\def\labelenumi{\arabic{enumi}.}
\setcounter{enumi}{-1}
\tightlist
\item
  Obtenga un \emph{token} de GitHub en la siguiente opción de menú de su
  perfil de usuario: \emph{Settings - Developer settings - Personal
  access tokens}. Seleccione las operaciones de tipo ``repo''. Copie el
  \emph{token} en un lugar seguro, ya que lo necesitará para
  autenticarse en GitHub.
\item
  Realice un \emph{fork} a su cuenta en GitHub del repositorio
  localizado en la dirección
  \url{https://github.com/pf0953-programacionr/2022-ii-tutorial-git-repo-ejemplo}.
  Obtendrá un repositorio llamado
  ``https://github.com/{[}nombre-usuario{]}/2022-ii-tutorial-git-repo-ejemplo'',
  en donde {[}nombre-usuario{]} es su nombre de usuario en GitHub.
\item
  Con la opción \emph{File - New Project - Version Control - Git} de
  RStudio, clone a su computadora el repositorio que acaba de bifurcar.
\item
  Con el editor de RStudio, abra el archivo \texttt{README.md}, agregue
  una línea y guarde el archivo.
\item
  Luego, ejecute los siguientes comandos desde la la ventana
  \emph{Terminal} de RStudio. Nota: las líneas que empiezan con
  \texttt{\#} son comentarios.
\end{enumerate}

\begin{Shaded}
\begin{Highlighting}[]
\NormalTok{\# a. Parámetros de configuración: nombre y dirección de correo del usuario.}
\NormalTok{\#    Debe cambiar [email{-}usuario] y [nombre{-}usuario] por sus propios datos.}
\NormalTok{git config {-}{-}global user.email [email{-}usuario]}
\NormalTok{git config {-}{-}global user.name [nombre{-}usuario]}
\NormalTok{\# Para revisar los parámetros de configuración:}
\NormalTok{git config {-}{-}global {-}{-}list}

\NormalTok{\# b. Revisión de los archivos con modificaciones.}
\NormalTok{git status}

\NormalTok{\# c. Adición (add) de los archivos modificados al "área de espera".}
\NormalTok{\#    El punto (.) indica que se agregarán todos los archivos modificados.}
\NormalTok{git add .}

\NormalTok{\# d. Grabado (commit) del conjunto de archivos modificados,}
\NormalTok{\#    junto con un mensaje explicativo:}
\NormalTok{git commit {-}m "Agregar línea 2"}

\NormalTok{\# e. "Subida" (push) de las modificaciones al repositorio remoto.}
\NormalTok{\#    En este paso, es posible que deba utilizar su nombre de usuario/clave}
\NormalTok{\#    o su token de GitHub para autenticarse.}
\NormalTok{git push}
\end{Highlighting}
\end{Shaded}

\begin{enumerate}
\def\labelenumi{\arabic{enumi}.}
\setcounter{enumi}{4}
\tightlist
\item
  Revise los cambios aplicados en el repositorio remoto en GitHub.
\item
  Agregue más líneas al archivo del repositorio local y sincronícelo con
  el remoto, realizando nuevamente los pasos del b al e para cada
  \emph{commit}. Recuerde que los comentarios de cada \texttt{commit}
  deben reflejar los cambios que están siendo aplicados.
\end{enumerate}

\hypertarget{recursos-de-interuxe9s-1}{%
\section{Recursos de interés}\label{recursos-de-interuxe9s-1}}

\emph{Git}. (s. f.). Recuperado 28 de agosto de 2022, de
\url{https://git-scm.com/}

\emph{GitHub Archive Program}. (s. f.). GitHub Archive Program.
Recuperado 10 de abril de 2022, de
\url{https://archiveprogram.github.com/}

Higgins, M. (s. f.). \emph{Reproducible Templates for Analysis and
Dissemination}. Coursera. Recuperado 11 de abril de 2022, de
\url{https://www.coursera.org/learn/reproducible-templates-analysis}

Klavins, S. (2020). \emph{Version Control part 1}. Medium.
\url{https://stevenklavins94.medium.com/version-control-part-1-c5f1b43127f6}

\bookmarksetup{startatroot}

\hypertarget{referencias-bibliogruxe1ficas}{%
\chapter*{Referencias
bibliográficas}\label{referencias-bibliogruxe1ficas}}
\addcontentsline{toc}{chapter}{Referencias bibliográficas}

\hypertarget{refs}{}
\begin{CSLReferences}{1}{0}
\leavevmode\vadjust pre{\hypertarget{ref-gandrud_reproducible_2020}{}}%
Gandrud, Christopher. 2020. \emph{Reproducible Research with {R} and
{RStudio}}. Third edition. The {R} Series. Boca Raton, FL: CRC Press.

\leavevmode\vadjust pre{\hypertarget{ref-longley_geographic_2005}{}}%
Longley, Paul A., Michael F. Goodchild, David J. Maguire, and David W.
Rhind. 2005. \emph{Geographic {Information} {Systems} and {Science}}.
2nd edition. Chichester ; Hoboken, NJ: Wiley.

\leavevmode\vadjust pre{\hypertarget{ref-olaya_sistemas_2020}{}}%
Olaya, Víctor. 2020. {``Sistemas de {Información} {Geográfica}.''}
\url{https://volaya.github.io/libro-sig/}.

\leavevmode\vadjust pre{\hypertarget{ref-peng_reproducible_2011}{}}%
Peng, Roger D. 2011. {``Reproducible {Research} in {Computational}
{Science}.''} \emph{Science} 334 (6060): 1226--27.
\url{https://doi.org/10.1126/science.1213847}.

\leavevmode\vadjust pre{\hypertarget{ref-singleton_establishing_2016}{}}%
Singleton, Alex David, Seth Spielman, and Chris Brunsdon. 2016.
{``Establishing a Framework for {Open} {Geographic} {Information}
Science.''} \emph{International Journal of Geographical Information
Science} 30 (8): 1507--21.
\url{https://doi.org/10.1080/13658816.2015.1137579}.

\end{CSLReferences}



\end{document}
